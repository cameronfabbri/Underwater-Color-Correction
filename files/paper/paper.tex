
%% bare_conf.tex
%% V1.4
%% 2012/12/27
%% by Michael Shell
%% See:
%% http://www.michaelshell.org/
%% for current contact information.
%%
%% This is a skeleton file demonstrating the use of IEEEtran.cls
%% (requires IEEEtran.cls version 1.8 or later) with an IEEE conference paper.
%%
%% Support sites:
%% http://www.michaelshell.org/tex/ieeetran/
%% http://www.ctan.org/tex-archive/macros/latex/contrib/IEEEtran/
%% and
%% http://www.ieee.org/

%%*************************************************************************
%% Legal Notice:
%% This code is offered as-is without any warranty either expressed or
%% implied; without even the implied warranty of MERCHANTABILITY or
%% FITNESS FOR A PARTICULAR PURPOSE! 
%% User assumes all risk.
%% In no event shall IEEE or any contributor to this code be liable for
%% any damages or losses, including, but not limited to, incidental,
%% consequential, or any other damages, resulting from the use or misuse
%% of any information contained here.
%%
%% All comments are the opinions of their respective authors and are not
%% necessarily endorsed by the IEEE.
%%
%% This work is distributed under the LaTeX Project Public License (LPPL)
%% ( http://www.latex-project.org/ ) version 1.3, and may be freely used,
%% distributed and modified. A copy of the LPPL, version 1.3, is included
%% in the base LaTeX documentation of all distributions of LaTeX released
%% 2003/12/01 or later.
%% Retain all contribution notices and credits.
%% ** Modified files should be clearly indicated as such, including  **
%% ** renaming them and changing author support contact information. **
%%
%% File list of work: IEEEtran.cls, IEEEtran_HOWTO.pdf, bare_adv.tex,
%%                    bare_conf.tex, bare_jrnl.tex, bare_jrnl_compsoc.tex,
%%                    bare_jrnl_transmag.tex
%%*************************************************************************

% *** Authors should verify (and, if needed, correct) their LaTeX system  ***
% *** with the testflow diagnostic prior to trusting their LaTeX platform ***
% *** with production work. IEEE's font choices can trigger bugs that do  ***
% *** not appear when using other class files.                            ***
% The testflow support page is at:
% http://www.michaelshell.org/tex/testflow/



% Note that the a4paper option is mainly intended so that authors in
% countries using A4 can easily print to A4 and see how their papers will
% look in print - the typesetting of the document will not typically be
% affected with changes in paper size (but the bottom and side margins will).
% Use the testflow package mentioned above to verify correct handling of
% both paper sizes by the user's LaTeX system.
%
% Also note that the "draftcls" or "draftclsnofoot", not "draft", option
% should be used if it is desired that the figures are to be displayed in
% draft mode.
%
\documentclass[conference]{IEEEtran}
% Add the compsoc option for Computer Society conferences.
%
% If IEEEtran.cls has not been installed into the LaTeX system files,
% manually specify the path to it like:
% \documentclass[conference]{../sty/IEEEtran}

% *** PDF, URL AND HYPERLINK PACKAGES ***
%
\usepackage{url}
% url.sty was written by Donald Arseneau. It provides better support for
% handling and breaking URLs. url.sty is already installed on most LaTeX
% systems. The latest version and documentation can be obtained at:
% http://www.ctan.org/tex-archive/macros/latex/contrib/url/
% Basically, \url{my_url_here}.

% correct bad hyphenation here
\hyphenation{op-tical net-works semi-conduc-tor}

\begin{document}

\title{UGAN: Underwater Image Restoration using Generative Adversarial Networks}

\author{\IEEEauthorblockN{Cameron Fabbri}
\IEEEauthorblockA{Information Directorate\\
Air Force Research Laboratory \\
Rome, NY, USA.\\
cameron.fabbri@us.af.mil}
\and
\IEEEauthorblockN{Md Jahidul Muslim}
\IEEEauthorblockA{UMN\\}
\and
\IEEEauthorblockN{Junaed Sattar}
\IEEEauthorblockA{UMN\\
}}

% for over three affiliations, or if they all won't fit within the width
% of the page, use this alternative format:
% 
%\author{\IEEEauthorblockN{Michael Shell\IEEEauthorrefmark{1},
%Homer Simpson\IEEEauthorrefmark{2},
%James Kirk\IEEEauthorrefmark{3}, 
%Montgomery Scott\IEEEauthorrefmark{3} and
%Eldon Tyrell\IEEEauthorrefmark{4}}
%\IEEEauthorblockA{\IEEEauthorrefmark{1}School of Electrical and Computer Engineering\\
%Georgia Institute of Technology,
%Atlanta, Georgia 30332--0250\\ Email: see http://www.michaelshell.org/contact.html}
%\IEEEauthorblockA{\IEEEauthorrefmark{2}Twentieth Century Fox, Springfield, USA\\
%Email: homer@thesimpsons.com}
%\IEEEauthorblockA{\IEEEauthorrefmark{3}Starfleet Academy, San Francisco, California 96678-2391\\
%Telephone: (800) 555--1212, Fax: (888) 555--1212}
%\IEEEauthorblockA{\IEEEauthorrefmark{4}Tyrell Inc., 123 Replicant Street, Los Angeles, California 90210--4321}}

% make the title area
\maketitle

% As a general rule, do not put math, special symbols or citations
% in the abstract
\begin{abstract}
Autonomous underwater robots often rely on visual input for decision making due to its non-intrusive and passive nature
. However, due to many factors such as light refraction, particles in the water, and color distortion, images are often times very noisy. This paper propose a method using Generative Adversarial Networks (GANs) to denoise underwater images, and show that these images provide both increased accuracy
for an underwater tracking algorithm, as well as a more visually appealing image. Furthermore, we show how recently
proposed methods are able to generate a dataset for the purpose of underwater image reconstruction.
\end{abstract}

\IEEEpeerreviewmaketitle

\section{Introduction}
Vision is a commonly used sensor in autonomous underwater robots due to its non-intrusive, passive, and energy
effecient nature. Despite these advantages, many underwater environments can be quite noisy due to light
refraction, blue or green hues, and particles present in the water. Much of this noise comes from color distortion,
where images have green or blue haze which may cause difficulty in tasks such as segmentation, tracking, or
classification. Algorithms that have been shown to work well on images not underwater may fail due to their
indirect or direct use of color. While there have been many very successful recent approaches towards colorization
\cite{bao2017cvae}, all are focused on the task of grayscale to color. 


\section{Related Work}



\section{Method}

\subsection{Generative Adversarial Networks}



\subsection{Objective}



\subsection{Network Architectures}



\subsubsection{Generator}



\subsubsection{Discriminator}



\section{Conclusion}



\section*{Acknowledgment}

%
%\begin{thebibliography}{1}
%\bibitem{hires}{Johnson‐Roberson, Matthew, et al. "High‐Resolution Underwater Robotic Vision‐Based Mapping and Three‐Dimensional Reconstruction for Archaeology." Journal of Field Robotics 34.4 (2017): 625-643.}
%
%\bibitem{col1}{Zhang, Richard, Phillip Isola, and Alexei A. Efros. "Colorful image colorization." European Conference on Computer Vision. Springer International Publishing, 2016.}

%\bibitem{col2}{Cheng, Zezhou, Qingxiong Yang, and Bin Sheng. "Deep colorization." Proceedings of the IEEE International Conference on Computer Vision. 2015.}

%\bibitem{col3}{Iizuka, Satoshi, Edgar Simo-Serra, and Hiroshi Ishikawa. "Let there be color!: joint end-to-end learning of global and local image priors for automatic image colorization with simultaneous classification." ACM Transactions on Graphics (TOG) 35.4 (2016): 110.}

%\bibitem{pix2pix}{Isola, Phillip, et al. "Image-to-image translation with conditional adversarial networks." arXiv preprint arXiv:1611.07004 (2016).}

%\bibitem{watergan}{Li, Jie, et al. "WaterGAN: Unsupervised Generative Network to Enable Real-time Color Correction of Monocular Underwater Images." arXiv preprint arXiv:1702.07392 (2017).}

%\end{thebibliography}

%\bibliography{/home/fabbric/Research/colorCorrection/files/paper/cambibs}
\bibliography{cambibs}
\bibliographystyle{ieeetr}

% that's all folks
\end{document}


